\documentclass[10pt,twocolumn]{article}

\usepackage[margin=30pt]{geometry}
\usepackage{amsmath}
\usepackage{physics}
\usepackage{amssymb}
\usepackage{amsfonts}
\usepackage{hyperref}
\usepackage{thmbox}
\usepackage{geometry}
\usepackage{graphicx}


\newcommand{\R}{\mathbb{R}}
\newcommand{\C}{\mathbb{C}}
\newcommand{\Z}{\mathbb{Z}}

\newtheorem[L]{thm}{Theorem}[section]

\author{Haiyang Wang, Leslie Greengard, Nils Jan Fredrik, Sam Potter}
\date{\today}
\title{Fast Solver for Stokes Flow in 2D}

\begin{document}

\maketitle

\begin{abstract}
  In this paper, we exploited the \textit{return to Poiseuille} phenomenon to build a solver for the interior plane Stokes flow with a domain that is union of \textit{standard pieces}. 
  Each standard piece's outlets are required to be long enough straight pipes, so that the return to Poiseuille is guaranteed. 
  This allows us to build the solver for each standard pieces with Poiseuille boundary condition, 
  and then connect the solution of each standard pieces based on the zero-flux 
  and singular-valued-ness of pressure to get the solution of the global domain. 

  With prebuilt solvers for standard pieces, we can connect them to form an arbitrarily large and complex domain. 
  Combining the local solvers would then only take $O(N^2)$ time, where $N$ is the number of standard pieces. 
  Much faster than solving the global problem directly.
\end{abstract}

\section{Introduction}

For plane Stokes flow, the biharmonic equation formulation are well known 
and developed within theory of complex variable from last century \cite{ladyzhenskayaMathematicalTheoryViscous1964}.
Various numerical schemes, such as boundary integral equation(BIE) and rational function approximation, 
have been developed accordingly \cite{greengardIntegralEquationMethods1996,trefethenApproximationTheoryApproximation2019}. 


The \textit{return to Poiseuille} phenomenon, or \textit{Saint-Venant's principle} for plane elasticity, are well-established 
from the last century \cite{coRecentDevelopmentsConcerning1983,gregoryTractionBoundaryValue1980,horganDECAYESTIMATESBIHARMONIC1989}. 
In particular, in a straight pipe with arbitrary incoming flow, the differences of Stokes flow and Poiseuille flow would decay exponentially fast. 
Quickly the flow would be indistinguishable from Poiseuille flow within machine-precision. 
Therefore it is a good numerical hypothesis to assume that the flow is Poiseuille at where the flow is far from non-straight parts 
of the pipe.

In this paper, we use the BIM from \cite{greengardIntegralEquationMethods1996} 
and exploited the \textit{return to Poiseuille} hypothesis to build solvers for multiple standard pieces. 
These standard pieces can be connect to form a large domain of pipe network, 
which can be solved by simply combining the solvers of the standard pieces. 
The BIM is coupled with the biharmonic Fast Multiple Method (FMM) 
to reduce space and time complexity for the matrix-vector product \cite{FlatironinstituteFmm2d2022}.
Directly evaluating the BIM's solution near the boundary is known to be very inaccurate.
Thus, we have adopted the methods from 
\cite{wuSolutionStokesFlow2020,helsingEvaluationLayerPotentials2008} 
to correctly evaluate of layer potentials near the boundary. 


This paper is organized as follows. In Section \ref{mathprelim}, we define the Stokes boundary value problem, 
the corresponding biharmonic boundary value problem, and then the integral equation of it. 
We also mention the analytic evidence for the \textit{return to Poiseuille} hypothesis. In Section \ref{sec:numericalmethod}, we presents the Nyestorm discretization
of the integral equation. 
The numerical experiments of connecting standard pieces and numerical evidence for \textit{return to Poiseuille}
hypothesis are contained in Section \ref{sec:numericalresults}, 
followed by conclusions in Section \ref{sec:conclusions}.

\section{Mathematical Preliminaries\label{mathprelim}}

In this section, we first state the Stokes equation, translate it into the biharmonic equation, 
and then derive the Boundary Integral equation. This whole derivation is nothing new from \cite{greengardIntegralEquationMethods1996}. 
Then, we will present the analytic bound for \textit{return to Poiseuille} \cite{gregoryTractionBoundaryValue1980}. 

\subsection{Stokes Boundary Value Problem}

The plane linear Stokes equations are
\begin{align}
  \nu \Delta u = \frac 1 \rho \pdv{p}{x},\quad &\nu \Delta v = \frac 1\rho \pdv{p}{y} 
  \label{stokes} \\
  \pdv{u}{x} + \pdv{v}{y} &= 0
  \label{continuity}
\end{align}
where $u,v$ are components of velocity, 
$\rho$ is the density, 
$\nu$ is the viscosity, 
and $p$ is the pressure. 
Additionally, vorticity is defined as $\zeta  = u_y - v_x$. 

We are interested in interior boundary value problem on a finite $(M+1)$-ply connected domain $D\subset \mathbb R^2$,
with boundary $\partial D =  \Gamma = \Gamma_0 \cup \Gamma_1 \cup \cdots \cup \Gamma_M$, 
where $\Gamma_0$ is the exterior boundary, and $\Gamma_1,\cdots, \Gamma_M$ are the interior boundaries. We restrict our attention to 
problems where the velocity is given on the boundary:

\begin{align}
  u = h_2(t),\quad v = - h_1(t), \quad t\in \Gamma
  \label{bdr-velocity}
\end{align}

\subsection{The Biharmonic Potential}

\paragraph*{Biharmonic Stream Function.} $(\ref{continuity})$ implies the existence of the stream function $W(x,y)$ such that:
\begin{align}
  \pdv{W}{x} = -v,\quad \pdv{W}{y} = u \label{stream-1}
\end{align}

Following (\ref{stokes},\ref{continuity}), it is easy to see that the stream function satisfies the biharmonic equation (\ref{biharmonic}),
and the boundary velocity conditions ($\ref{bdr-velocity}$) can be understood as the boundary conditions for the biharmonic equation (\ref{bih-bv}):
\begin{align}
  &\Delta^2 W(x,y) = \Delta \zeta = 0 \label{biharmonic}\\
  &\pdv{W}{x}(t) = h_1(t), \pdv{W}{y}(t) = h_2(t), \quad t\in \Gamma\label{bih-bv}
\end{align}

\paragraph*{Goursat's Formula.} It's been long established that any plane biharmonic function $W(x,y)$ can be expressed by Goursat's formula 
\begin{align}
  W(x,y) = \Re (\bar z \phi(z) + \chi (z)) \label{Goursat}
\end{align}
where $\phi, \chi$ are analytic functions of complex variable $z = x+yi$. In the following, we will be identifying $(x,y) \in \mathbb{R}^2$ with $x + yi\in \mathbb{C}$
as a convenient abuse of notation. 


The Muskhelishvili's formula
\begin{align}
    u(x,y) + iv(x,y) 
    &= \phi(z) + z \overline{\phi'(z)} + \overline{\psi(z)}
    \label{muskhelishvili}
\end{align}, where $\psi = \chi'$,  transforms the biharmonic boundary condition \eqref{bih-bv} into 
\begin{align}
  \phi(t) + t\overline{\phi'(t)} + \overline{\psi(t)} 
  = h(t), \quad
  t \in \Gamma
\end{align} where $h(t) =  h_1(t) + ih_2(t)$,  and $t$ is understood as a complex variable. 

For Stokes equation, simple calculation leads to
\begin{align}
  \zeta + \frac{i}{\nu}p = 4\phi'(z) \label{pressure-and-vorticity}
\end{align}allowing us to evaluate pressure and vorticity once we solved the biharmonic BVP.

\paragraph*{Sherman-Lauricella Representation.} The boundary integral equation is an ansatz 
based on of an extension of Sherman-Lauricella representation 
proposed in \cite{greengardIntegralEquationMethods1996}.

\begin{align}
  \phi(z) &=
    \frac {1}{2\pi i} \int_\Gamma \frac{\omega(\xi)}{\xi - z} d\xi  
    + \sum_{k=1}^M C_k \log (z-z_k)
    \\
  \psi(z) &=
    \frac {1}{2\pi i} \int_\Gamma \frac{\overline{\omega(\xi)}d\xi +  \omega(\xi)\overline{d\xi}}{\xi - z}  
    - \frac {1}{2\pi i} \int_\Gamma \frac{\overline{\xi} \omega(\xi)}{(\xi - z)^2} d\xi  
    \\
    & \quad + \sum _{k=1}^M 
    \left( \frac{b_k}{z-z_k} + \overline C_k \log (z-z_k) -  C_k \frac{\overline z_k}{z-z_k} \right) \nonumber 
\end{align}
where $\omega$ is an unknown complex density to be solved on $\Gamma$, 
$z_k$ are arbitrarily prescribed point inside the component curves $\Gamma_k$, 
and $C_k, b_k$ are constants defined by 
\begin{align}
  C_k = \int_{\Gamma_k} \omega(\xi) |d\xi|, \quad b_k = 2 \Im\int_{\Gamma_k} \overline{\omega(\xi)} {d\xi}
\end{align}

\paragraph*{Boundary Integral Equation.} 
Letting a point $z$ in the interior of $D$ approach to a point on the boundary $t\in \Gamma$, 
the classical formulae for the limiting values of Cauchy-type integral 
gives us the an integral equation for $\omega$:
\begin{align}
  \omega(t) 
  &+ \frac 1{2\pi i} \int_{\Gamma} \omega(\xi) d \ln \frac{\xi - t}{\overline{\xi - t}} - \frac 1{2\pi i} \int_\Gamma \overline{\omega(\xi)} d \ln \frac{\xi - t}{\overline{\xi - t}} \\
  &+ \sum_{k=1}^M \left( \frac{\bar b_k}{\bar t- \bar z_k} +  2C_k \log |t-z_k| + \bar C_k \frac{t-z_k}{\bar t - \bar z_k} \right) \nonumber\\
  &+ \frac{\overline b_0}{\bar t - \bar z^*} \nonumber \\
  &= h(t)
\end{align}
the extra term $\frac{\overline b_0}{\bar t - \bar z^*}$ vanishes when the zero-net-flux condition $\Re \int_\Gamma \bar h(t) dt = 0$ is satisfied. 
The invertibility of this integral equation is similar 
to the standard proof of invertibility for elasticity problems \cite{muskhelishviliBasicProblemsMathematical1977}, hence omitted.

\subsection{Return to Poiseuille\label{sec:ret2poi}} 

On the domain of a semi-infinite pipe $D_L = \{(x,y)\mid x \ge 0, |y| \le L\}$, with the boundaries 
\begin{align}
  \Gamma_L &=&& \Gamma_L^1\quad\quad\quad& \cup &\Gamma_L^2 &\cup&\Gamma_L^3 \\
  &=&&\{(0,y)||y| \le L \}& \cup &\{(x,L)|x\ge 0\} &\cup& \{(x,-L)\mid x\ge 0\}\nonumber
\end{align}
where $\Gamma_L^2,\Gamma_L^3$ are walls with the non-slippery boundary conditions, 
and $\Gamma_L^1$ is the only part with non-zero boundary conditions. 
Return to Poiseuille means that regardless of the boundary conditions on $\Gamma_L^1$,
the flow's profile at $x = l$ will converge Poiseuille flow as $l$ approaches to infinity. Without lost of generality, assuming there is zero net flux at $\Gamma_L^1$,
return to Poiseuille is equivalent to return to zero flow. 

The biharmonic BVP can then written as 

\begin{align}
  \pdv{W}{x}(x,y) = \pdv{W}{y}(x,y) = W(x,y) = 0, \quad &(x,y) \in \Gamma_L^2 \cup \Gamma_L^3 \\
  \pdv{W}{x}(0,y) = f(y),\quad \pdv{W}{y}(0,y) = g(y), \quad &|y| \le L
\end{align}
where are some functions with $f(\pm L) = g(\pm L) = \int_{-L}^L g(y)dy = 0$. 


This biharmonic BVP is identical to the "self-equilibrated" traction BVP in the theory of elasticity studied in
\cite{gregoryTractionBoundaryValue1980,horganDECAYESTIMATESBIHARMONIC1989,coRecentDevelopmentsConcerning1983}. 
When $f''',g'''$ are of bounded variation, 
this problem has a unique solution spanned by the Papkovich-Fadle eigenfunctions \cite{gregoryTractionBoundaryValue1980}.
The first eigenfunction is dominated by $e^{-xk/2L}$, where $$k \simeq 4.2$$ 
is the smallest positive real parts of the roots 
of the transcendental equation $\sin^2\lambda - \lambda^2=0$. 
This gives the decay rate of return to Poiseuille hypothesis. This agrees with [numerical experiment to be included later]. 
% TODO

% should I explain more about how to make use of the return to poiseuille hypothesis here? 


\section{Description of the Numerical Method\label{sec:numericalmethod}}

\subsection{Geometry of the Boundary}

adaptive panel refinement, smooth cap and corners, etc.


\subsection{Integral equation}

\section{Numerical Results and Discussion\label{sec:numericalresults}}

\subsection{numerical evidence of return to poiseuille}
\subsection{the error of combining local to global}
\subsection{a complicated network of pipes to show the power of this method}

\section{Conclusions\label{sec:conclusions}}

\subsection{summarize what I've done}
\subsection{outlook. What other work might be followed?}
\bibliographystyle{plain}
\bibliography{applie-math}
\end{document}
