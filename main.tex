\documentclass[10pt,twocolumn,letterpaper]{article}

\usepackage[margin=35pt]{geometry}
\usepackage{amsmath}
\usepackage{physics}
\usepackage{amssymb}
\usepackage{amsfonts}
\usepackage{hyperref}
\usepackage{thmbox}
\usepackage{geometry}
\usepackage{graphicx}
\usepackage[]{subcaption}
\usepackage[inline]{enumitem}

\newcommand{\R}{\mathbb{R}}
\newcommand{\C}{\mathbb{C}}

\renewcommand{\thefootnote}{\fnsymbol{footnote}}

\author{Haiyang Wang, Nils Jan Fredrik Fryklund, Samuel Potter, Leslie Greengard}
\date{\today}
\title{A Fast Solver for Stokes Flow in 2D: [we need a better title]}

\begin{document}



\maketitle

\begin{abstract}
  In this paper, we exploit the \textit{return to Poiseuille} phenomenon:
  a Stokes flow would quickly develop to the Poiseuille flow along a straight channel.
  This allows us to quickly solve the interior plane Stokes equation
  on a domain that is a union of \textit{standard pieces}.
  Each standard piece is a pipe with inlets/outlets
  being long enough straight channels, such that when two standard pieces are connecting,
  where they connect is in middle of a long
  straight channel\footnote{
    The length of straight channel is greater than 7 times of the width,
    as indicated by Figure~\ref{fig:r2pnumerical}.}, hence the flow
  at the connection would be close to Poiseuille flow within machine precision.
  Then, instead of solving stokes equation for the global domain,
  we can solve the Stokes equation
  for each standard pieces with boundary condition of Poiseuille velocity profile at inlets/outlets,
  and easily interface these local solutions to build a solution for the global domain.

  Once the Stokes equation with boundary conditions of Poiseuille velocity
  profile is pre-solved on each standard piece, the standard pieces can be
  connected to form a complex domain of channel network. Interfacing the
  solutions of standard pieces would instantly give a high-order accurate
  solution of Stokes equation for the global domain. For example, in
  Figure~\ref{fig:connection-error}, interfacing the local solutions took only
  0.3 seconds, while directly solving on the global domain took 24 minutes.
\end{abstract}

\begin{figure*}[t]
  \centering
  \includegraphics[width=\textwidth]{pic/connection-error-rough.png}
  \caption{
    Solutions of the Stokes equation in a complex channel geometry 
    as a union of 22 pieces of 7 kinds of standard pieces.
    The first sub-figure is a color-plot of magnitude of velocity field inside the
    domain, with colorbar on its right. The black lines in the
    first sub-figure marks the boundary of each standard piece.
    The other three sub-figures is color plot of absolute differences
    between the connected solution and the global solution in pressure,
    vorticity, and velocity field in the $\log_{10}$ scale, with colorbar on on the right. 
    Each standard pieces are solved
    with required accuracy of $10^{-12}$ and the global domain is solved with
    required accuracy of $10^{-10}$.}\label{fig:connection-error}

\end{figure*}

\section{Introduction}

The \textit{return to Poiseuille} phenomenon, or \textit{Saint-Venant's
  principle} in the theory of plane elasticity, are well-established from the
last century~\cite{coRecentDevelopmentsConcerning1983,gregoryTractionBoundaryValue1980,horganDECAYESTIMATESBIHARMONIC1989}.
To be more specific, in a straight channel with laminar and incompressible
incoming flow, the flow would quickly converge into a Poiseuille flow toward
the outlet: the difference of the flow and the Poiseuille flow would decay at
an exponential rate for Stokes flow. Therefore it is a good numerical
hypothesis to assume that the flow is Poiseuille in middle of a lone straight
channel, regardless of the incoming and outgoing flow.

For plane Stokes flow, the biharmonic equation formulation are well known and
developed within theory of complex variable from the last century
~\cite{ladyzhenskayaMathematicalTheoryViscous1964}. Various numerical schemes,
such as boundary integral equation (BIE) and rational function approximation,
have been developed accordingly
~\cite{greengardIntegralEquationMethods1996,trefethenApproximationTheoryApproximation2019}.
In this paper, we use the biharmonic BIE formulation for the plane Stokes
equation from~\cite{greengardIntegralEquationMethods1996} to solve the Stokes
equation on several standard pieces with Poiseuille boundary condition at
inlets/outlets.
% The biharmonic BIE is coupled with a Fast Multiple Method (FMM)
% for 2D biharmonic equation to reduce the time and space complexity of solving
% the BIE \cite{FlatironinstituteFmm2d2022}.
% Directly evaluating the BIE's solution near the boundary could be numerically
% unstable as the integral is nearly-singular. Thus, we have adopted the methods
% from \cite{wuSolutionStokesFlow2020,helsingEvaluationLayerPotentials2008} for
% accurate evaluation of layer potentials near the boundary.

The main idea of this paper is to apply \textit{Return to Poiseuille} as a high
order accurate numerical hypothesis. It allows us to pre-solve a few standard
pieces which have inlets and outlets being long enough straight channels, and
are with boundary condition of Poiseuille velocity profile. Once the
pre-solved, the \textit{Return to Poiseuille} hypothesis allows us to interface
solutions of Stokes flow on each standard pieces to get a solution for any
complex channel networks that is a union of the standard pieces. Interfacing is
by solving a system of linear equations, based on the constraints of
zero-net-flux and continuity of pressure. This system of linear equations can
be solved instantly and accurately.

This paper is organized as follows. In Section~\ref{mathprelim}, we define the
Stokes boundary value problem, the corresponding biharmonic boundary value
problem, and then the integral equation of it. We also mention the analytic
evidence and predicted exponential convergence rate for the \textit{return to
  Poiseuille} in a semi-infinite straight channel. Then, we explain how to
interface the local solutions of standard pieces by an simple example. In
Section~\ref{sec:numericalmethod}, we presents the Nystr\"om discretization of
the integral equation, which is solved iteratively by Generalized Minimal
Residual Method (GMRES).
% At the end of this section, we also points to the 
% reference of other techniques we have adopted in this paper: 
% accurate evaluation of nearly singular integral, 
% GMRES with removed nullspace, 
% the smoothing of corners of the geometry,
% and the biharmonic FMM. 
The numerical experiments of connecting standard pieces and numerical evidence
for \textit{return to Poiseuille} hypothesis are contained in Section~\ref{sec:numericalresults}, 
followed by conclusions and possible further work
in Section~\ref{sec:conclusions}.

\section{Mathematical Preliminaries\label{mathprelim}}

In this section, we briefly review the plane Stokes equation, its biharmonic
form, and its biharmonic boundary integral equation. More detailed discussion
can be found in~\cite{greengardIntegralEquationMethods1996}. Then, we will
present an analytic estimate for the exponential decay rate of \textit{return
  to Poiseuille} hypothesis~\cite{gregoryTractionBoundaryValue1980}, and explain
how we have applied it as a numerical hypothesis: how to pre-solve each
standard pieces and how to interface the pre-solved solutions for a domain of
connected standard pieces.

\subsection{Boundary Integral Equation}

\paragraph{Stokes Boundary Value Problem.}

Recall that the plane Stokes equations are~\cite{ladyzhenskayaMathematicalTheoryViscous1964}:
\begin{align}
  \nu \Delta u = \frac 1 \rho \pdv{p}{x},\quad & \nu \Delta v = \frac 1\rho \pdv{p}{y}
  \label{stokes}                                                                       \\
  \pdv{u}{x} + \pdv{v}{y}                      & = 0
  \label{continuity}
\end{align}
where $u,v$ are components of velocity, $p$ is the pressure,
$\rho$ and $\nu$ are the density and viscosity, which are constants.
Another important physics quantity, vorticity, is defined as $\zeta  = u_y - v_x$.

We are interested in Dirichlet boundary value problem (BVP) of Stokes equation
on a bounded $(M+1)$-ply connected domain $D\subset \mathbb R^2$, with boundary
$\partial D = \Gamma = \Gamma_0 \cup \Gamma_1 \cup \cdots \cup \Gamma_M$, where
$\Gamma_0$ is the exterior boundary, and $\Gamma_1,\ldots, \Gamma_M$ are the
interior boundaries. On the boundary $\Gamma$, the velocity is defined by given
functions $h_1,h_2$:
\begin{align}
  u = h_2(t),\quad v = - h_1(t), \quad t\in \Gamma
  \label{bdr-velocity}
\end{align}
For the specific purpose of this paper,
the boundary velocity profile is zero everywhere except at the inlets/outlets
of channels, where a Poiseuille velocity profile is specified.

\paragraph*{Biharmonic Equation.} $(\ref{continuity})$ implies the existence of the
stream function $W(x,y)$ such that~\cite{greengardIntegralEquationMethods1996}:
\begin{align}
  \pdv{W}{x} = -v,\quad \pdv{W}{y} = u \label{stream-1}
\end{align}

Following (\ref{stokes},\ref{continuity}), it is easy to see that the stream
function satisfies the biharmonic equation (\ref{biharmonic}), and the
Dirichlet BVP ($\ref{bdr-velocity}$) can be understood as the following
biharmonic BVP:
\begin{align}
   & \Delta^2 W(x,y) = \Delta \zeta = 0,                        & (x,y)\in D \label{biharmonic} \\
   & \pdv{W}{x}(t) = h_1(t),\quad \pdv{W}{y}(t) = h_2(t), \quad & t\in \Gamma\label{bih-bv}
\end{align}
% where $h_1,h_2$ are from equation (\ref{bdr-velocity}).

\paragraph*{Goursat's Formula.} It has been long established that any plane
biharmonic function $W(x,y)$ can be expressed by Goursat's formula~\cite{muskhelishviliBasicProblemsMathematical1977}:
\begin{align}
  W(x,y) = \Re (\bar z \phi(z) + \chi (z)) \label{Goursat}
\end{align}
where the Goursat's functions $\phi, \chi$ are analytic functions of complex variable $z = x+yi$.
In the following, we will be identifying $(x,y) \in \mathbb{R}^2$ with $x + yi \in \mathbb{C}$.

Velocity, pressure, and vorticity can be conveniently expressed with the
Goursat's functions. The Muskhelishvili's formula~\eqref{muskhelishvili}
expresses velocity field and another formula~\eqref{pressure-and-vorticity}
gives the pressure and vorticity~\cite{muskhelishviliBasicProblemsMathematical1977}:
\begin{align}
  -v + ui                & = \pdv{W}{x} + i\pdv{W}{y}
  = \phi(z) + z \overline{\phi'(z)} + \overline{\psi(z)}
  \label{muskhelishvili}                                              \\
  \zeta + \frac{i}{\nu}p & = 4\phi'(z) \label{pressure-and-vorticity}
\end{align} where $\psi = \chi'$.

The biharmonic boundary value problem (\ref{biharmonic},\ref{bih-bv}), using
the Muskhelishvili's formula (\ref{muskhelishvili}), can be rewritten as
\begin{align}
  \phi(t) + t\overline{\phi'(t)} + \overline{\psi(t)}
  = h(t), \quad
  t \in \Gamma \label{musk-bvp}
\end{align} where $h(t) =  h_1(t) + ih_2(t)$,  and $t$ is understood as a complex variable.

\paragraph*{Sherman-Lauricella Representation.} The Sherman-Lauricella
Representation proposes a specific form of the Goursat's functions~\cite{greengardIntegralEquationMethods1996}. 
And one can use this representation as an ansatz for BIE of the Biharmonic BVP~\eqref{musk-bvp}. The
Sherman-Lauricella representation is formulated as follows:
\begin{align}
  \phi(z) & =
  \frac {1}{2\pi i} \int_\Gamma \frac{\omega(\xi)}{\xi - z} d\xi
  + \sum_{k=1}^M C_k \log (z-z_k) \label{sl-phi}
  \\
  \psi(z) & =
  \frac {1}{2\pi i} \int_\Gamma \frac{\overline{\omega(\xi)}d\xi +  \omega(\xi)\overline{d\xi}}{\xi - z}
  - \frac {1}{2\pi i} \int_\Gamma \frac{\overline{\xi} \omega(\xi)}{{(\xi - z)}^2} d\xi  \label{sl-psi}
  \\
          & \quad + \sum _{k=1}^M
  \left( \frac{b_k}{z-z_k} + \overline C_k \log (z-z_k) -  C_k \frac{\overline z_k}{z-z_k} \right) \nonumber
\end{align}
where $\omega$ is an unknown complex density on $\Gamma$ to be solved,
$z_k$ are arbitrarily prescribed point inside the component curves $\Gamma_k$,
and $C_k, b_k$ are constants defined by:
\begin{align}
  C_k = \int_{\Gamma_k} \omega(\xi) |d\xi|, \quad b_k = 2 \Im\int_{\Gamma_k} \overline{\omega(\xi)} {d\xi}
\end{align}

It is worthnoting that although $\psi, \phi$ might be multiple-valued, 
the velocity, pressure, and vorticity are single-valued functions of $z$.

\paragraph*{Boundary Integral Equation.} Plugging the Sherman-Lauricella
representation (\ref{sl-phi},\ref{sl-psi}) into equation (\ref{musk-bvp}), and
letting a point $z$ in the interior of $D$ approach to a point on the boundary
$t\in \Gamma$ in~\eqref{musk-bvp}, the classical formulae for the limiting
values of Cauchy-type integral gives us the following integral equation for
$\omega$
~\cite{muschelisviliSingularIntegralEquations1972,greengardIntegralEquationMethods1996}:
\begin{align}
  \omega(t)
   & + \frac 1{2\pi i} \int_{\Gamma} \omega(\xi) d\ln \frac{\xi - t}{\overline{\xi - t}} - \frac 1{2\pi i} \int_\Gamma \overline{\omega(\xi)} d \frac{\xi - t}{\overline{\xi - t}} \label{bie} \\
   & + \sum_{k=1}^M \left( \frac{\bar b_k}{\overline{t- z_k}} +  2C_k \log |t-z_k| + \overline{C_k} \frac{t-z_k}{\overline{ t - z_k}} \right) \nonumber                                        \\
   & + \frac{\overline b_0}{\overline{ t - z^*}} \nonumber                                                                                                                                     \\
   & = h(t) \nonumber
\end{align}
the extra term $\frac{\overline b_0}{\overline{t - z^*}}$ vanishes 
when the zero-net-flux condition $\Re \int_\Gamma \bar h(t) dt = 0$ is satisfied, 
hence will be omitted in the Nystr\"om discretization of~\eqref{bie} later.
The invertibility of this integral equation is similar
to the standard proof of invertibility for elasticity problems~\cite{muskhelishviliBasicProblemsMathematical1977,greengardIntegralEquationMethods1996}, 
and are omitted.

\subsection{Return to Poiseuille\label{sec:ret2poi}}

In this section, we will first show the analytic estimate for the
\textit{return to Poiseuille} phenomenon, which is based on eigenfunction
analysis on a domain of a semi-infinite straight channel from the theory of
plane elasticity~\cite{gregoryTractionBoundaryValue1980}. Then, we explain how
to apply the \textit{return to Poiseuille} hypothesis, and how to interface the
solutions on standard pieces.

\paragraph*{Analytic Estimate for Return to Poiseuille. }

On the domain of a semi-infinite straight channel $D_L = \{(x,y)\mid x \ge 0, |y| \le L\}$,
with the boundaries
\begin{align}
  \Gamma_L & = \Gamma_L^1 \cup \Gamma_L^2 \cup \Gamma_L^3                                    \\
           & =\{(0,y)||y| \le L \} \cup \{(x,L)|x\ge 0\} \cup \{(x,-L)\mid x\ge 0\}\nonumber
\end{align}
where $\Gamma_L^2,\Gamma_L^3$ are walls with the non-slippery boundary conditions,
and $\Gamma_L^1$ is the inlet with boundary condition of an
incoming laminar incompressible flow.
\textit{Return to Poiseuille} means that regardless of the boundary velocity profile on $\Gamma_L^1$,
the flow's profile at $x = l$ will converge Poiseuille flow of same flux as $l\to\infty$.

Without lost of generality, assume there is zero net flux across $\Gamma_L^1$.
Then, \textit{return to Poiseuille} is equivalent to return to the zero flow,
i.e.\ the flows velocity profile at the vertical cross-section $x=l$ would
converge to zero at an exponential decay rate as $l\to\infty$. The BVP 
can be formulated as the following:
\begin{align}
   & \pdv{W(x,y)}{y}  = W(x,y) = 0,                    & (x,y) & \in \Gamma_L^2 \cup \Gamma_L^3                         \\
   & \pdv{W(0,y)}{x}  = f(y),\ \pdv{W(0,y)}{y} = g(y), & (0,y) & \in \Gamma_L^1  \label{eq:velocity-condition-at-inlet}
\end{align}
where $f,g$ are smooth enough functions satisfying 
$f(\pm L) = g(\pm L) = \int_{-L}^L g(y)dy = 0$,
so that the net-flux is zero and the boundary condition is continuous.

This biharmonic BVP is identical to the self-equilibrated traction BVP in the
theory of elasticity studied in~\cite{gregoryTractionBoundaryValue1980,horganDECAYESTIMATESBIHARMONIC1989,coRecentDevelopmentsConcerning1983}.
When $f^{\prime\prime\prime},g^{\prime\prime\prime}$ exist and are of bounded variation, this problem has a unique
solution spanned by the Papkovich-Fadle eigenfunctions~\cite{gregoryTractionBoundaryValue1980}. 
The absolute value of first
eigenfunction is dominated by $e^{-xk/2L}$, where
\begin{equation*}
  k \simeq 4.2  \footnote{This is the smallest positive real parts of the roots
    of the transcendental equation $\sin^2\lambda - \lambda^2=0$.}
\end{equation*}
This gives the decay rate of return to Poiseuille hypothesis,
which agrees with our numerical experiment in Figure \ref{fig:r2pnumerical}.

\paragraph{Return to Poiseuille as a Numerical Hypothesis.}
Given the analytic estimate above, it is easy to see that in a straight channel with
length greater than 8 times of the channel width, we can expect the flow to be
Poiseuille with 14th digits of accuracy at the outlet regardless of the
velocity profile on the inlet. Therefore, it is appropriate to require the
inlets/outlets of the standard pieces to be such straight channels, and assign
the Poiseuille boundary conditions on the inlets/outlets.

Figure~\ref{fig:connection-error} is a numerical example where the interfaced
solution is compared with a global solution, and high order accuracy is
achieved in both pressure, velocity, and vorticity. It is worth-noting that no
significant numerical error is observed at where the standard pieces are
connected.

\paragraph{Interface the Solutions on Standard Pieces.}

\begin{figure}[h!]
  \centering
  \includegraphics[width=0.4\textwidth]{pic/standard_pipe_flows_demo.png}
  \caption{The 2 generating flows for a Y-shaped standard piece: 
  the color inside the domain denotes the magnitude of the velocity of the flow, 
  the blue arrow denotes the boundary condition of unit-flux Poiseuille velocity profile at the inlets/outlets.}\label{fig:y_2_flows}
\end{figure}


For a standard piece with a total number of $m$ inlets/outlets, 
the boundary conditions of our interest, Poiseuille at the 
inlets/outlets and non-slippery elsewhere, can be simply
characterized by the flux of the Poiseuille flow at each inlet/outlet. The fluxes need 
to sum to zero, so the boundary conditions is a $m-1$ dimensional space. 
As the Stokes equation is linear, this means we only need to solve for 
$m-1$ flows on the standard piece, and superimpose these flows would give us 
any flow in the standard piece with boundary condition of Poiseuille at inlets/outlet 
and non-slippery elsewhere.
For example, in Figure~\ref{fig:y_2_flows}, a standard piece of Y shaped standard pieces, 
with a total number  of 3 inlets/outlets,
are pre-solved with 2 generating flows. 



With the the generating flows pre-solved for each standard piece, 
interfacing these flows is reduce to finding 
appropriate fluxes of the generating flows of the standard pieces such that:
\begin{enumerate}
  \item The fluxes matches at where the point of connection\label{cond:flux-match}.
  \item The pressure is a continuous function across the domain.\label{cond:pressure}
\end{enumerate}
The second condition is needed only when the domain of connected standard pieces 
have cycles, i.e.\ the domain is not simply connected. 

To demonstrate how these two conditions are turned into 
a linear system of equations for interfacing local solutions, 
let us consider the specific example in Figure~\ref{fig:interface_problem_0}, where the global domain 
is two connected Y-shaped standard pieces. 
The global domain is given with flux $f_{-1} = -1$ (incoming) at the left inlet 
and flux $f_{-2}=1$ (outgoing) at the right outlet. 
Each standard piece has two fluxes of generating flows need to be solved for, 
so there are total number of $4$ fluxes $f_0,f_1,f_2,f_3$ needs to be solved for. 

This problem can be abstracted into a graph theory problem as in Figure~\ref{fig:interface_problem_1},
where the the nodes $V_0,V_1,V_2,V_3$ denotes the interfaces among the standard pieces 
and global boundary conditions of fluxes, 
The edge $E_{0,1},E_{0,2},E_{31},E_{32}$ represents the generating flows 
with the fluxes $f_0,f_1,f_2,f_3$ to be solved for. 
The fluxes $f_{-1},f_{-2}$ are given on superficial edges representing the boundary fluxes of the global domain.

\begin{figure}[h!]
  \centering
  \begin{subfigure}[b]{0.45\textwidth}
    \centering
    \includegraphics[width=\textwidth]{pic/simple-interface-problem.png}
    \caption{The interface problem}\label{fig:interface_problem_0}
  \end{subfigure}
  \begin{subfigure}[b]{0.45\textwidth}
    \centering
    \includegraphics[width=\textwidth]{pic/simple-interface-problem-network.png}
    \caption{The abstract network for the interface problem}\label{fig:interface_problem_1}
  \end{subfigure}
  \caption{A simple interface problem with 2 Y-shaped standard pieces.}\label{fig:simple-interface-problem}
\end{figure}

The interface condition~\ref{cond:flux-match} of agreeing fluxes at the interfaces, 
in this graph theory setting, can be understood as the following: 
at each node, its adjacent edges have fluxes sum to zero. Requiring this to be true at all nodes gives us
the following linear system of equations:
\begin{align}
  \begin{pmatrix}
    1 & 1 & 0 & 0\\
    1 & 0 & 1 & 0\\
    0 & 1 & 0 & 1\\
    0 & 0 & 1 & 1\\
  \end{pmatrix}
  \begin{pmatrix}
    f_0\\
    f_1\\
    f_2\\
    f_3\\
  \end{pmatrix}
  =
  \begin{pmatrix}
    -f_{-1}\\
    0\\
    0\\
    -f_{-2}\\
  \end{pmatrix}
  =
  \begin{pmatrix}
    1\\
    0\\
    0\\
    -1\\
  \end{pmatrix} \label{eq:interface-flux-match}
\end{align}
This matrix has rank 1 deficiency. 
This rank deficiency is due to there is 1 cycle in the graph. 
This rank deficiency can be resolved by the continuity of pressure condition~\ref{cond:pressure}. 

Continuity of pressure, or single valued-ness of pressure, 
can be understood as that the pressure $p_i$ at each node $V_i$ is 
a well-defined single-value function. 
This condition would be automatically satisfied by the flux condition~\ref{cond:flux-match} 
if the global domain is simply connected. 
However, for a domain with cycles, 
such as the cycle $(V_0,V_1,V_3,V_2)$ in Figure~\ref{fig:interface_problem_1}, 
single-valued-ness of pressure can be understood as 
the pressure drops along the cycle would sum to zero. Therefore, we can gain a new equation: 
\begin{align}
  p_{10} + p_{31} + p_{23} + p_{02} =0 \label{eq:pressure-cycle}
\end{align}
where $p_{ij}$ is the pressure drop from node $V_j$ to node $V_i$. 
The pressure drops $p_{ij}$ depends linearly on the fluxes of generating flows of the standard pieces, 
For this case, $p_{10}, p_{02}$ are linear functions of $f_0,f_1$, 
and $p_{31}, p_{23}$ are linear functions of $f_2,f_3$. 
These linear functions can be computed in the pre-solving process,
and for this example, they are
\begin{align}
  \begin{pmatrix} p_{10} \\ p_{02} \end{pmatrix} 
  &= \begin{pmatrix}
    -46.02 & -14.78 \\
    14.78 & 46.02
  \end{pmatrix}
  \begin{pmatrix}
    f_0\\
    f_1
  \end{pmatrix} \label{eq:pd1} \\
  \begin{pmatrix} p_{31} \\ p_{23} \end{pmatrix} 
  &= \begin{pmatrix}
    46.02 & 14.78 \\
    -14.78 & -46.02
  \end{pmatrix}
  \begin{pmatrix}
    f_2\\
    f_3
  \end{pmatrix}\label{eq:pd2}
\end{align}

Combining equations~(\ref{eq:pressure-cycle}),~(\ref{eq:pd1}) and~(\ref{eq:pd2}), we have:
\begin{align}
  (46.02-14.78)(-f_0 + f_1 + f_2 - f_3) = 0
\end{align}
which completes~\eqref{eq:interface-flux-match} to a full rank linear system of equations 
and with the solutions $(f_0,f_1,f_2,f_3) = (0.5,0.5,-0.5,-0.5)$. 

The process of interfacing local solutions in the example of Figure~\ref{fig:simple-interface-problem} 
can be generalized into a algorithm for interfacing of any connected standard pieces. 
The key idea of the algorithm is rather simple: 
generating the appropriate linear equations of matching fluxes and matching pressure drops,  
and then solving for fluxes of generating flows.
However, it involves troublesome bookkeeping of the correspondence between the graph theory problem 
and the generating fluxes and pressure-drops of each standard pieces, 
so the detailed and complete description of the algorithm is omitted here. 
One can find source code for this algorithm at \url{https://github.com/WangHaiYang874/stokes2d}. 

\paragraph{Numerical Stability of Interface Algorithm}
The numerical stability of the connecting algorithm can be analyzed as follows. 
The condition of matching fluxed~\ref{cond:flux-match} is translated to the linear equation~\eqref{eq:interface-flux-match},
which contains only $1,0$ and therefore no inaccuracy is introduced. 
The only inaccuracy is introduced by computing of the matrices in~\eqref{eq:pd1} and~\eqref{eq:pd2} in the pre-solving process. 
Each entries of the matrices of pressure drops can be computed with required accuracy of $\epsilon=10^{-12}$, Therefore ... 


\section{Description of Numerical Methods\label{sec:numericalmethod}}
In this section,
we will first present Nystr\"om discretization of boundary integral equation (\ref{bie}).
And we will briefly explain the smoothing of the geometry and accurate near boundary evaluation of layer potential.

\subsection{Boundary Integral Equation}

The boundary curve $\Gamma_k$ is given by the parametrization $\Gamma_k = \{
  t^k(a): a\in \left[A_k,A_{k+1}\right]\}$, and discretized into $N_k$ points
$t^k_i = t^k(a^k_i)$. Associate to each point $t^k_j$ are the unknown complex
density $\omega^k_j$, the derivative $d^k_j = t^{k\prime}(a^k_j)$, and the
quadrature weight $w^k_j$. In total, we have $N= \sum_{k=0}^M N_k$ points. The
Nystr\"om discretization of BIE (\ref{bie}) is:
\begin{align}
  \omega_j^k
  + \sum_{m=0}^{M}\sum_{n=1}^{N_k} K_1(t^k_j,t^m_n) \omega^k_j
  + \sum_{m=0}^{M}\sum_{n=1}^{N_k} K_2(t^k_j,t^m_n) \overline{\omega^k_j} = h^k_j
  \label{nystrom}
\end{align} where $h^k_j = h(t^k_j)$ and the kernels $K_1, K_2$ are given by
\begin{align}
  K_1(t^k_j, t^m_n)
   & = \frac{w^m_n}{\pi} \Im (\frac{d^m_n}{t^m_n-t^k_j}) + K_1^s(t^k_j,t^m_n)                                          \\
  K_2(t^k_j, t^m_n)
   & = \frac{w^m_n}{\pi} \frac{\Im((t^m_n-t^k_j)\overline{d^m_n})}{(\overline{t^m_n - t^k_j)^2}}  + K_2^s(t^k_j,t^m_n)
\end{align}
with $K_1^s, K_2^s$ representing the singular sources are:
\begin{align}
  K_1^s(t^k_j,t^m_n) & = \delta_m w^m_n \left(\frac{i\overline{d^m_n}}{\overline{t^k_j - z_m}}
  + 2 \log |t^k_j - z_m| \right)                                                               \\
  K_2^s(t^k_j,t^m_n) & = \delta_{m}w^m_n \frac{t^k_j-z_m-id^m_n}{\overline{t^k_j - z_m}}
\end{align}
where $\delta_m = 1$ excepts for $\delta_0 = 0$. In the limiting case of $t^k_j = t^m_n$, the value of $K_1,K_2$ are:
\begin{align}
  K_1(t^k_j, t^k_j) & = \frac{w^k_j \kappa^k_j|d^k_j|}{2\pi} + K_1^s(t^k_j,t^k_j)          \\
  K_2(t^k_j, t^k_j) & = -\frac{w^k_j\kappa^k_j(d^k_j)^2}{2\pi|d^k_j|} + K_2^s(t^k_j,t^k_j)
\end{align}where $\kappa^k_j$ is the signed curvature at the point $t^k_j$.

The Nystr\"om discretizetion \eqref{nystrom} can be written more compactly as
the following matrix equation:
\begin{align}
  \omega + K_1\omega + K_2\overline{\omega} = h \label{nys-mateq}
\end{align}
This equation is separated into real and imaginary parts, and iteratively solved by the GMRES \cite{saadGMRESGeneralizedMinimal1986}.
For each GMRES iteration, evaluating the left hand side of \eqref{nys-mateq}
is required, and evaluating by a dense matrix-vector product would require space and time complexity of $O(N^2)$. Too expensive when $N$ is large. Therefore, we used a biharmonic fmm provided by the Flatiron Institute instead, which would only have time and space complexity of $O(N)$ \cite{FlatironinstituteFmm2d2022}.

The matrix equation \eqref{nys-mateq} obeys the zero-net-flux condition,
therefore only has solution when $\Re \int_\Gamma \overline{h(t)} dt = 0$. This
also means that \eqref{nys-mateq} has rank deficiency, which might cause GMRES
converging slowly. This issue can be avoided by adding a double layer term

Evaluation of the layer potentials near boundary is done as in
\cite{wuSolutionStokesFlow2020}.

\subsection{Geometry of the Boundary}

The key to spectral convergence of GMRES is to have smooth boundary, or for
piecewise smooth boundary, one can use special treatment as in
\cite{wuSolutionStokesFlow2020} to ensure the spectral convergence is
preserved. Here for this paper, we focused on smooth geometry. We adopted the
ideas from
\cite{epsteinSmoothedCornersScattered2016,baggeHighlyAccurateSpecial2021} to
smooth the corners of the boundary by convolution, and added superficial caps
at the inlets and outlets. [insert a figure here]. The geometry is adaptively
discretized into Gauss-Legendre panels, as described in
\cite{wuSolutionStokesFlow2020}.

\section{Numerical Results and Discussion\label{sec:numericalresults}}
\begin{figure}[h!]
  \centering
  \begin{subfigure}[b]{0.4\textwidth}
    \centering
    \includegraphics[width=\textwidth]{pic/rtp_cv.png}
    \caption{Numerical Convergence rate of return to Poiseuille flow in a straight channel}
    \label{fig:rtp_cv}
  \end{subfigure}
  \begin{subfigure}[b]{0.4\textwidth}
    \centering
    \includegraphics[width=\textwidth]{pic/rtppipe.png}
    \caption{$\log_{10}$ of the absolute value of the velocity, vorticity, and pressure within the straight channel.}
    \label{fig:rtppipe}
  \end{subfigure}
  \caption{Numerical exponential rate for return to Poiseuille flow.
    This is solution of Stokes BVP on a straight channel of length 10 and width 1,
    with non-slippery boundary condition on the top and bottom walls,
    an incoming flow (smooth and randomly generated) of zero-net-flux on the left inlet,
    and no outgoing flow on the right outlet.
    (a) The semilogy of magnitude of velocity, pressure, and vorticity
    along each vertical cross section along the channel.
    % This agrees with the predicted converge rate in Section \ref{sec:ret2poi} all the way to 14th digits of accuracy. 
    (b) The color plot of the magnitude of velocity, pressure, and vorticity in $\log_{10}$ scale in the straight channel.
  }
  \label{fig:r2pnumerical}
\end{figure}

\subsection{Numerical evidence of return to poiseuille}

The numerical evidence for return to Poiseuille phenomenon is demonstrated on a
straight pipe of width $1$ and length $8$ as in Figure \ref{fig:r2pnumerical}.
On the left boundary, a smooth velocity profile is imposed. This velocity
profile is an arbitrarily picked smooth function that satisfies the requirement
of equation (\ref{eq:velocity-condition-at-inlet}). On the rest of the curve,
non-slippery condition is imposed.

Figure \ref{fig:rtp_cv} shows that the rate of returning to the zero flow is
agreed with the predicted rate from Section \ref{sec:ret2poi} up to 14th digits
of accuracy. Figure \ref{fig:rtppipe} is a color plot where the color indicates
the $\log_{10}$ of absolute value of the velocity, pressure, and vorticity.

\subsection{a complicated network of pipes to show the power of this method}

\section{Conclusions\label{sec:conclusions}}

\section{Acknowledgements\label{sec:acknowledgements}}

We thank Charles S. Peskin and Manas Rachh for many useful discussions
pretaining to this work. We thank Manas Rachh and Libin Lu for providing
support for the Flatiron Institute's FMM2D library
\cite{FlatironinstituteFmm2d2022}.

\subsection{summarize what I've done}
\subsection{outlook. What other work might be followed?}
\bibliographystyle{plain}
\bibliography{references}

\end{document}